% Options for packages loaded elsewhere
\PassOptionsToPackage{unicode}{hyperref}
\PassOptionsToPackage{hyphens}{url}
\PassOptionsToPackage{dvipsnames,svgnames,x11names}{xcolor}
%
\documentclass[
]{article}

\usepackage{amsmath,amssymb}
\usepackage{iftex}
\ifPDFTeX
  \usepackage[T1]{fontenc}
  \usepackage[utf8]{inputenc}
  \usepackage{textcomp} % provide euro and other symbols
\else % if luatex or xetex
  \usepackage{unicode-math}
  \defaultfontfeatures{Scale=MatchLowercase}
  \defaultfontfeatures[\rmfamily]{Ligatures=TeX,Scale=1}
\fi
\usepackage{lmodern}
\ifPDFTeX\else  
    % xetex/luatex font selection
    \setmainfont[]{Latin Modern Roman}
  \setmathfont[]{Latin Modern Math}
\fi
% Use upquote if available, for straight quotes in verbatim environments
\IfFileExists{upquote.sty}{\usepackage{upquote}}{}
\IfFileExists{microtype.sty}{% use microtype if available
  \usepackage[]{microtype}
  \UseMicrotypeSet[protrusion]{basicmath} % disable protrusion for tt fonts
}{}
\makeatletter
\@ifundefined{KOMAClassName}{% if non-KOMA class
  \IfFileExists{parskip.sty}{%
    \usepackage{parskip}
  }{% else
    \setlength{\parindent}{0pt}
    \setlength{\parskip}{6pt plus 2pt minus 1pt}}
}{% if KOMA class
  \KOMAoptions{parskip=half}}
\makeatother
\usepackage{xcolor}
\setlength{\emergencystretch}{3em} % prevent overfull lines
\setcounter{secnumdepth}{5}
% Make \paragraph and \subparagraph free-standing
\makeatletter
\ifx\paragraph\undefined\else
  \let\oldparagraph\paragraph
  \renewcommand{\paragraph}{
    \@ifstar
      \xxxParagraphStar
      \xxxParagraphNoStar
  }
  \newcommand{\xxxParagraphStar}[1]{\oldparagraph*{#1}\mbox{}}
  \newcommand{\xxxParagraphNoStar}[1]{\oldparagraph{#1}\mbox{}}
\fi
\ifx\subparagraph\undefined\else
  \let\oldsubparagraph\subparagraph
  \renewcommand{\subparagraph}{
    \@ifstar
      \xxxSubParagraphStar
      \xxxSubParagraphNoStar
  }
  \newcommand{\xxxSubParagraphStar}[1]{\oldsubparagraph*{#1}\mbox{}}
  \newcommand{\xxxSubParagraphNoStar}[1]{\oldsubparagraph{#1}\mbox{}}
\fi
\makeatother


\providecommand{\tightlist}{%
  \setlength{\itemsep}{0pt}\setlength{\parskip}{0pt}}\usepackage{longtable,booktabs,array}
\usepackage{calc} % for calculating minipage widths
% Correct order of tables after \paragraph or \subparagraph
\usepackage{etoolbox}
\makeatletter
\patchcmd\longtable{\par}{\if@noskipsec\mbox{}\fi\par}{}{}
\makeatother
% Allow footnotes in longtable head/foot
\IfFileExists{footnotehyper.sty}{\usepackage{footnotehyper}}{\usepackage{footnote}}
\makesavenoteenv{longtable}
\usepackage{graphicx}
\makeatletter
\def\maxwidth{\ifdim\Gin@nat@width>\linewidth\linewidth\else\Gin@nat@width\fi}
\def\maxheight{\ifdim\Gin@nat@height>\textheight\textheight\else\Gin@nat@height\fi}
\makeatother
% Scale images if necessary, so that they will not overflow the page
% margins by default, and it is still possible to overwrite the defaults
% using explicit options in \includegraphics[width, height, ...]{}
\setkeys{Gin}{width=\maxwidth,height=\maxheight,keepaspectratio}
% Set default figure placement to htbp
\makeatletter
\def\fps@figure{htbp}
\makeatother
% definitions for citeproc citations
\NewDocumentCommand\citeproctext{}{}
\NewDocumentCommand\citeproc{mm}{%
  \begingroup\def\citeproctext{#2}\cite{#1}\endgroup}
\makeatletter
 % allow citations to break across lines
 \let\@cite@ofmt\@firstofone
 % avoid brackets around text for \cite:
 \def\@biblabel#1{}
 \def\@cite#1#2{{#1\if@tempswa , #2\fi}}
\makeatother
\newlength{\cslhangindent}
\setlength{\cslhangindent}{1.5em}
\newlength{\csllabelwidth}
\setlength{\csllabelwidth}{3em}
\newenvironment{CSLReferences}[2] % #1 hanging-indent, #2 entry-spacing
 {\begin{list}{}{%
  \setlength{\itemindent}{0pt}
  \setlength{\leftmargin}{0pt}
  \setlength{\parsep}{0pt}
  % turn on hanging indent if param 1 is 1
  \ifodd #1
   \setlength{\leftmargin}{\cslhangindent}
   \setlength{\itemindent}{-1\cslhangindent}
  \fi
  % set entry spacing
  \setlength{\itemsep}{#2\baselineskip}}}
 {\end{list}}
\usepackage{calc}
\newcommand{\CSLBlock}[1]{\hfill\break\parbox[t]{\linewidth}{\strut\ignorespaces#1\strut}}
\newcommand{\CSLLeftMargin}[1]{\parbox[t]{\csllabelwidth}{\strut#1\strut}}
\newcommand{\CSLRightInline}[1]{\parbox[t]{\linewidth - \csllabelwidth}{\strut#1\strut}}
\newcommand{\CSLIndent}[1]{\hspace{\cslhangindent}#1}

\usepackage{arxiv}
\usepackage{orcidlink}
\usepackage{amsmath}
\usepackage[T1]{fontenc}
\makeatletter
\@ifpackageloaded{caption}{}{\usepackage{caption}}
\AtBeginDocument{%
\ifdefined\contentsname
  \renewcommand*\contentsname{Table of contents}
\else
  \newcommand\contentsname{Table of contents}
\fi
\ifdefined\listfigurename
  \renewcommand*\listfigurename{List of Figures}
\else
  \newcommand\listfigurename{List of Figures}
\fi
\ifdefined\listtablename
  \renewcommand*\listtablename{List of Tables}
\else
  \newcommand\listtablename{List of Tables}
\fi
\ifdefined\figurename
  \renewcommand*\figurename{Figure}
\else
  \newcommand\figurename{Figure}
\fi
\ifdefined\tablename
  \renewcommand*\tablename{Table}
\else
  \newcommand\tablename{Table}
\fi
}
\@ifpackageloaded{float}{}{\usepackage{float}}
\floatstyle{ruled}
\@ifundefined{c@chapter}{\newfloat{codelisting}{h}{lop}}{\newfloat{codelisting}{h}{lop}[chapter]}
\floatname{codelisting}{Listing}
\newcommand*\listoflistings{\listof{codelisting}{List of Listings}}
\makeatother
\makeatletter
\makeatother
\makeatletter
\@ifpackageloaded{caption}{}{\usepackage{caption}}
\@ifpackageloaded{subcaption}{}{\usepackage{subcaption}}
\makeatother
\ifLuaTeX
  \usepackage{selnolig}  % disable illegal ligatures
\fi
\usepackage{bookmark}

\IfFileExists{xurl.sty}{\usepackage{xurl}}{} % add URL line breaks if available
\urlstyle{same} % disable monospaced font for URLs
\hypersetup{
  pdftitle={The Name of the Title Is Hope},
  pdfauthor={Max Gabler; Wanshu Jiang; Christoph Kiesl; Leonard Pöhls; Alexander Rieber; Ansgar Scherp},
  colorlinks=true,
  linkcolor={blue},
  filecolor={Maroon},
  citecolor={Blue},
  urlcolor={Blue},
  pdfcreator={LaTeX via pandoc}}

\newcommand{\runninghead}{A Preprint }
\title{The Name of the Title Is Hope}
\def\asep{\\\\\\ } % default: all authors on same column
\author{\textbf{Max
Gabler}\\\href{mailto:max.gabler@uni-ulm.de}{max.gabler@uni-ulm.de}\asep\textbf{Wanshu
Jiang}\\\href{mailto:wanshu.jiang@uni-ulm.de}{wanshu.jiang@uni-ulm.de}\asep\textbf{Christoph
Kiesl}\\\href{mailto:christoph.kiesl@uni-ulm.de}{christoph.kiesl@uni-ulm.de}\asep\textbf{Leonard
Pöhls}\\\href{mailto:leonard.poehls@uni-ulm.de}{leonard.poehls@uni-ulm.de}\asep\textbf{Alexander
Rieber}\\\href{mailto:alexander.rieber@uni-ulm.de}{alexander.rieber@uni-ulm.de}\asep\textbf{Ansgar
Scherp}\\\href{mailto:ansgar.scherp@uni-ulm.de}{ansgar.scherp@uni-ulm.de}}
\date{}
\begin{document}
\maketitle
\begin{abstract}
I am the abstract
\end{abstract}

\renewcommand*\contentsname{Table of contents}
{
\hypersetup{linkcolor=}
\setcounter{tocdepth}{3}
\tableofcontents
}
\section{1. Introduction}\label{introduction}

Business model innovation (BMI) is a key activity to maintain
competitiveness and even gain a competitive advantage (Pucihar et al.
2019; Teece 2018). It is therefore no surprise that the interest in BMI
and methods of measuring it has grown rapidly over the last twenty
years. Researchers have recently called for a BMI measurement instrument
that is more comprehensive and advanced than already existing ones
(Huang and Ichikohji 2023). The scale developed by Spieth \& Schneider
(2016) provides managers and practitioners with a measurement index for
business model innovativeness. This measurement model only validates
applicability of BMI theory (Huang and Ichikohji 2023) and is
insufficient for longitudinal studies (Clauss 2017). Hence, this measure
is not adequate for a time series analysis of BMI. Furthermore, it
refers only to BMI as new-to-the-firm and is not able to grasp BMI in
the sense of new-to-the-industry and new-to-market. Clauss (2017)
developed a similar measure with similar downsides.

We tackle this gap by measuring BMI as the similarity between business
descriptions of one company over the years. The Security and Exchange
Commission (SEC) requires US-based companies to submit so called 10-K
filings every year, where at one point, the company has to state a
description of its business. We utilize the BERTScore as a similarity
measure and calculate the similarity between these description for a
company for different years.

\begin{itemize}
\item
  Key findings and Contribution Our contribution is made in a number of
  ways. Firstly, we tackle two issues raised in the study by Lee \& Hong
  (2014): we employ a more reliable and contemporary methodology for
  extracting the business model (BM) from 10-K filings and we are able
  to extend the scope of their study. Secondly, we build on the concept
  of alternative industry classification put forth by Hoberg \& Phillips
  (2016) and propose an industry classification system based on a firm's
  BM. Thirdly, we propose a novel measure for BMI that is sufficient for
  longitudinal studies.
\item
  paragraph 4 What are 10-K filings? Describe the data source and our
  way of using the Data (only brief)
\item
  paragraph 5 (robustness checks)
\end{itemize}

In spite of the growing interest in BMI and the increasing number of
theoretical and empirical studies in this field, the research of BMI is
still in a preliminary state (Huang and Ichikohji 2023). Consequently,
there is considerable variation in the definitions of BMI, with some
definitions being more similar to one another than others (Foss and
Saebi 2017). Spieth \& Schneider (2016) identify three core dimensions a
company's BM is comprised of: its value proposition, its value creation
architecture and its revenue model logic. Based on this, BMI can be
conceptualized as a change that is new-to-the-firm in at least one of
these dimensions. Furthermore, Spieth and Schneider (2016) introduce a
measurement model to evaluate these three dimensions of BMI. They
develop an index by first specifying the contents, followed by a
specification of the indicators and assessing their content validity,
assessing the indicators collinearity and finally assessing the external
validity. A total of twelve indicators for measuring the innovativeness
of the BM were identified through a comprehensive literature review and
through engagement with industry practitioners. The external validity of
the formative indicators was successfully validated through a survey of
200 experts in strategy and innovation management (Spieth and Schneider
2016). Clauss (2017) employs a very similar approach. After specifying
the domain and dimensionality of BMI trough literature research, the
author divides his scale into three hierarchical levels consisting of 41
reflective items, 10 subconstructs and three main dimensions, which are
similar to the ones mentioned earlier. The scale was validated through
two samples from the manufacturing industry and further demonstrated
nomological validity (Clauss 2017). However, both measures are subject
to three significant limitations. Firstly, both measures lack a temporal
component. Consequently, they are inadequate for use in longitudinal
studies or ex-post evaluations of BMI. Secondly, BMI is only measured at
the new-to-the-firm level rather than at the new-to-the-industry or
new-to-the-market level. Thirdly, both measures rely on interviews and
questionnaires, which makes conducting large-scale studies
time-consuming and reliant on the willingness of the companies to
cooperate (Clauss 2017; Spieth and Schneider 2016). This study tackles
both the first and third limitation.

The process of text mining 10-K filings is not a novel concept. Hoberg
\& Phillips (2016) present a novel approach to defining industry
boundaries. This is achieved through the parsing of the product
descriptions provided by firm 10-K filings and creating word vectors.
Specifically, the authors identify and exclude proper nouns, which
include common words and geographic locations.They then create word
vectors for each firm and year, which enables the measurement of product
similarity over time. In this way the authors demonstrate shortcomings
in the traditional industry classification systems such as the Standard
Industry Classification (SIC) and the North American Industry
Classification System (NAICS), which are not able to account for
temporal changes. The new method is capable of capturing changes in
industry boundaries and competitor sets over time, thereby providing a
dynamic industry classification system. In their study, Lee \& Hong
(2014) examine the evolution of a firm's BM over time. The authors
represent each document as a vector of keywords, which is similar to the
approach utilized by Hoberg \& Phillips (2016). After identifying the
Item 1 part of the 10-K filings as the most crucial part for describing
a firm's BM, Lee \& Hong (2014) filter these for relevant sentences.
Subsequently, the authors construct keyword vectors, which represent the
concept of the BM. Therefore, the evolution of the BM is depicted as the
change in the distribution of keywords over time. Nevertheless, this
approach is not without shortcomings. The authors advocate for a more
robust methodology, such as incorporating multi-word phrases in the
keyword vectors, to enhance the reliability of the approach (Lee and
Hong 2014).

The rest of the paper proceeds as follows. Section 2 describes our
preprocessing and our data. Section 3 lays out the BERT-model and our
estimations. Section 4 dicusses our results, and Section 5 concludes our
study.

\section{2. Data and Descriptive
Statistics}\label{data-and-descriptive-statistics}

\subsection{Preprocessing with Gemini}\label{preprocessing-with-gemini}

As already mentioned, 10-K filings are in general very large text
documents, where Item 1 of these filings is no exception. Table 1 shows
the average, minimum and maximum length of the original Item 1 part in
our sample. In order to utilize all information regarding the BM in the
Item 1 part and pass the text to our BERT-model, we decided to let
Google's GenAI chatbot Gemini summarize them to a maximum length of 512
tokens. We inserted our prompt at the beginning of each text file and
passed it via an API to Gemini \footnote{We forked and used following
  Github Repository: https://github.com/skranz/gemini\_ex.}. We used
following prompt: ``Summarize the business model from the following
text. Answer with a continuous text and with five hundred twelve tokens
at max. Set your focus on sources of revenue, the intended customer
base, products, distribution channels and details of financing. Use only
information from the following the text''.\footnote{The spelling error
  in the last sentence of the prompt was found after processing the Item
  1. After evaluating the Summaries, this error did not cause any
  issues.} ``intended customer base'' and ``product'' refer to the value
offering, ``distribution channels'' refers to the value architecture,
and ``sources of revenue'' and ``details of financing'' refer to the
revenue model. Thus, every dimension of the definition of BMI by Spieth
\& Schneider (2016) is covered by this prompt. To check quality and
accuracy of the summaries by Gemini, we draw a random sample of 100
filings and compare the original text with the summary. More precise, we
first read the original file with a focus on the points mentioned in the
prompt above and then check, if the summary also contains these points.
A list of the sample with the summaries is in the Appendix.

\begin{itemize}
\tightlist
\item
  result of this check
\item
  explain why ``tokens'' not words in the prompt: mostly more tokens
  than words, BERT uses first 512 tokens. so to use the whole output of
  Gemini, text limit to 512 tokens
\end{itemize}

TODO - Descriptive Table1 for document length of original filings

\subsection{The Dataset}\label{the-dataset}

We collect 10-Ks filings from the digital SEC Database, using the
category ``10-K'' as extraction condition. Since the focus of our study
lies on company's BM, we only use the Item 1 part, since this is the
most crucial part of the 10-K filings for describing the companies BM
(Lee and Hong 2014).

//Our observations are limited to an intersection of such companies,
which on the one hand has been made available to the SEC since 2001 in a
publicly accessible list of 10.284 companies (Appendix), of which 7590
are listed (on stock exchange). On the other hand, we consider companies
that filed 10-K reports with the SEC between 2017 and 2023
//-\textgreater{} rewrite as step by step, how we got to the final list
of companies

We exclude companies from the financial sector, namely companies with a
SIC Code starting with six. We consider the filings from 2017 to 2023.

TODO

\begin{itemize}
\tightlist
\item
  Table2 like Alex suggested
\item
  Descriptive Table3 for document length of processed filings
\item
  Description of Table3 and the final Dataset
\end{itemize}

\section{3. Empirical Framework}\label{empirical-framework}

\subsection{BERT and BERTScore}\label{bert-and-bertscore}

\section{4. Results and Discussion}\label{results-and-discussion}

\section{5. Conclusion}\label{conclusion}

\section{Acknowledgement}\label{acknowledgement}

\begin{itemize}
\tightlist
\item
  Jonathan for IT Support
\item
  Prof.~Kranz for Repo
\end{itemize}

\section*{Appendix}\label{appendix}
\addcontentsline{toc}{section}{Appendix}

\phantomsection\label{refs}
\begin{CSLReferences}{1}{0}
\bibitem[\citeproctext]{ref-clauss_measuring_2017}
Clauss, Thomas. 2017. {``Measuring Business Model Innovation:
Conceptualization, Scale Development, and Proof of Performance.''}
\emph{R\&D Management} 47 (3): 385--403.
\url{https://doi.org/10.1111/radm.12186}.

\bibitem[\citeproctext]{ref-foss_fifteen_2017}
Foss, Nicolai J., and Tina Saebi. 2017. {``Fifteen {Years} of {Research}
on {Business} {Model} {Innovation}: {How} {Far} {Have} {We} {Come}, and
{Where} {Should} {We} {Go}?''} \emph{Journal of Management} 43 (1):
200--227. \url{https://doi.org/10.1177/0149206316675927}.

\bibitem[\citeproctext]{ref-hoberg_text-based_2016}
Hoberg, Gerard, and Gordon Phillips. 2016. {``Text-{Based} {Network}
{Industries} and {Endogenous} {Product} {Differentiation}.''}
\emph{Journal of Political Economy} 124 (5): 1423--65.
\url{https://doi.org/10.1086/688176}.

\bibitem[\citeproctext]{ref-huang_review_2023}
Huang, WenJun, and Takeyasu Ichikohji. 2023. {``A Review and Analysis of
the Business Model Innovation Literature.''} \emph{Heliyon} 9 (7):
e17895. \url{https://doi.org/10.1016/j.heliyon.2023.e17895}.

\bibitem[\citeproctext]{ref-lee_business_2014}
Lee, Jihwan, and Yoo S. Hong. 2014. {``Business {Model} {Mining}:
{Analyzing} a {Firm}'s {Business} {Model} with {Text} {Mining} of
{Annual} {Report}.''} \emph{Industrial Engineering and Management
Systems} 13 (4): 432--41.
\url{https://doi.org/10.7232/iems.2014.13.4.432}.

\bibitem[\citeproctext]{ref-pucihar_drivers_2019}
Pucihar, Andreja, Gregor Lenart, Mirjana Kljajić Borštnar, Doroteja
Vidmar, and Marjeta Marolt. 2019. {``Drivers and {Outcomes} of
{Business} {Model} {Innovation}---{Micro}, {Small} and {Medium}-{Sized}
{Enterprises} {Perspective}.''} \emph{Sustainability} 11 (2): 344.
\url{https://doi.org/10.3390/su11020344}.

\bibitem[\citeproctext]{ref-spieth_business_2016}
Spieth, Patrick, and Sabrina Schneider. 2016. {``Business Model
Innovativeness: Designing a Formative Measure for Business Model
Innovation.''} \emph{Journal of Business Economics} 86 (6): 671--96.
\url{https://doi.org/10.1007/s11573-015-0794-0}.

\bibitem[\citeproctext]{ref-teece_business_2018}
Teece, David J. 2018. {``Business Models and Dynamic Capabilities.''}
\emph{Long Range Planning} 51 (1): 40--49.
\url{https://doi.org/10.1016/j.lrp.2017.06.007}.

\end{CSLReferences}



\end{document}
