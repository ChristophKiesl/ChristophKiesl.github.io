% Options for packages loaded elsewhere
\PassOptionsToPackage{unicode}{hyperref}
\PassOptionsToPackage{hyphens}{url}
\PassOptionsToPackage{dvipsnames,svgnames,x11names}{xcolor}
%
\documentclass[
]{article}

\usepackage{amsmath,amssymb}
\usepackage{iftex}
\ifPDFTeX
  \usepackage[T1]{fontenc}
  \usepackage[utf8]{inputenc}
  \usepackage{textcomp} % provide euro and other symbols
\else % if luatex or xetex
  \usepackage{unicode-math}
  \defaultfontfeatures{Scale=MatchLowercase}
  \defaultfontfeatures[\rmfamily]{Ligatures=TeX,Scale=1}
\fi
\usepackage{lmodern}
\ifPDFTeX\else  
    % xetex/luatex font selection
    \setmainfont[]{Latin Modern Roman}
  \setmathfont[]{Latin Modern Math}
\fi
% Use upquote if available, for straight quotes in verbatim environments
\IfFileExists{upquote.sty}{\usepackage{upquote}}{}
\IfFileExists{microtype.sty}{% use microtype if available
  \usepackage[]{microtype}
  \UseMicrotypeSet[protrusion]{basicmath} % disable protrusion for tt fonts
}{}
\makeatletter
\@ifundefined{KOMAClassName}{% if non-KOMA class
  \IfFileExists{parskip.sty}{%
    \usepackage{parskip}
  }{% else
    \setlength{\parindent}{0pt}
    \setlength{\parskip}{6pt plus 2pt minus 1pt}}
}{% if KOMA class
  \KOMAoptions{parskip=half}}
\makeatother
\usepackage{xcolor}
\setlength{\emergencystretch}{3em} % prevent overfull lines
\setcounter{secnumdepth}{5}
% Make \paragraph and \subparagraph free-standing
\makeatletter
\ifx\paragraph\undefined\else
  \let\oldparagraph\paragraph
  \renewcommand{\paragraph}{
    \@ifstar
      \xxxParagraphStar
      \xxxParagraphNoStar
  }
  \newcommand{\xxxParagraphStar}[1]{\oldparagraph*{#1}\mbox{}}
  \newcommand{\xxxParagraphNoStar}[1]{\oldparagraph{#1}\mbox{}}
\fi
\ifx\subparagraph\undefined\else
  \let\oldsubparagraph\subparagraph
  \renewcommand{\subparagraph}{
    \@ifstar
      \xxxSubParagraphStar
      \xxxSubParagraphNoStar
  }
  \newcommand{\xxxSubParagraphStar}[1]{\oldsubparagraph*{#1}\mbox{}}
  \newcommand{\xxxSubParagraphNoStar}[1]{\oldsubparagraph{#1}\mbox{}}
\fi
\makeatother


\providecommand{\tightlist}{%
  \setlength{\itemsep}{0pt}\setlength{\parskip}{0pt}}\usepackage{longtable,booktabs,array}
\usepackage{calc} % for calculating minipage widths
% Correct order of tables after \paragraph or \subparagraph
\usepackage{etoolbox}
\makeatletter
\patchcmd\longtable{\par}{\if@noskipsec\mbox{}\fi\par}{}{}
\makeatother
% Allow footnotes in longtable head/foot
\IfFileExists{footnotehyper.sty}{\usepackage{footnotehyper}}{\usepackage{footnote}}
\makesavenoteenv{longtable}
\usepackage{graphicx}
\makeatletter
\def\maxwidth{\ifdim\Gin@nat@width>\linewidth\linewidth\else\Gin@nat@width\fi}
\def\maxheight{\ifdim\Gin@nat@height>\textheight\textheight\else\Gin@nat@height\fi}
\makeatother
% Scale images if necessary, so that they will not overflow the page
% margins by default, and it is still possible to overwrite the defaults
% using explicit options in \includegraphics[width, height, ...]{}
\setkeys{Gin}{width=\maxwidth,height=\maxheight,keepaspectratio}
% Set default figure placement to htbp
\makeatletter
\def\fps@figure{htbp}
\makeatother
% definitions for citeproc citations
\NewDocumentCommand\citeproctext{}{}
\NewDocumentCommand\citeproc{mm}{%
  \begingroup\def\citeproctext{#2}\cite{#1}\endgroup}
\makeatletter
 % allow citations to break across lines
 \let\@cite@ofmt\@firstofone
 % avoid brackets around text for \cite:
 \def\@biblabel#1{}
 \def\@cite#1#2{{#1\if@tempswa , #2\fi}}
\makeatother
\newlength{\cslhangindent}
\setlength{\cslhangindent}{1.5em}
\newlength{\csllabelwidth}
\setlength{\csllabelwidth}{3em}
\newenvironment{CSLReferences}[2] % #1 hanging-indent, #2 entry-spacing
 {\begin{list}{}{%
  \setlength{\itemindent}{0pt}
  \setlength{\leftmargin}{0pt}
  \setlength{\parsep}{0pt}
  % turn on hanging indent if param 1 is 1
  \ifodd #1
   \setlength{\leftmargin}{\cslhangindent}
   \setlength{\itemindent}{-1\cslhangindent}
  \fi
  % set entry spacing
  \setlength{\itemsep}{#2\baselineskip}}}
 {\end{list}}
\usepackage{calc}
\newcommand{\CSLBlock}[1]{\hfill\break\parbox[t]{\linewidth}{\strut\ignorespaces#1\strut}}
\newcommand{\CSLLeftMargin}[1]{\parbox[t]{\csllabelwidth}{\strut#1\strut}}
\newcommand{\CSLRightInline}[1]{\parbox[t]{\linewidth - \csllabelwidth}{\strut#1\strut}}
\newcommand{\CSLIndent}[1]{\hspace{\cslhangindent}#1}

\usepackage{booktabs}
\usepackage{longtable}
\usepackage{array}
\usepackage{multirow}
\usepackage{wrapfig}
\usepackage{float}
\usepackage{colortbl}
\usepackage{pdflscape}
\usepackage{tabu}
\usepackage{threeparttable}
\usepackage{threeparttablex}
\usepackage[normalem]{ulem}
\usepackage{makecell}
\usepackage{xcolor}
\usepackage{arxiv}
\usepackage{orcidlink}
\usepackage{amsmath}
\usepackage[T1]{fontenc}
\makeatletter
\@ifpackageloaded{caption}{}{\usepackage{caption}}
\AtBeginDocument{%
\ifdefined\contentsname
  \renewcommand*\contentsname{Table of contents}
\else
  \newcommand\contentsname{Table of contents}
\fi
\ifdefined\listfigurename
  \renewcommand*\listfigurename{List of Figures}
\else
  \newcommand\listfigurename{List of Figures}
\fi
\ifdefined\listtablename
  \renewcommand*\listtablename{List of Tables}
\else
  \newcommand\listtablename{List of Tables}
\fi
\ifdefined\figurename
  \renewcommand*\figurename{Figure}
\else
  \newcommand\figurename{Figure}
\fi
\ifdefined\tablename
  \renewcommand*\tablename{Table}
\else
  \newcommand\tablename{Table}
\fi
}
\@ifpackageloaded{float}{}{\usepackage{float}}
\floatstyle{ruled}
\@ifundefined{c@chapter}{\newfloat{codelisting}{h}{lop}}{\newfloat{codelisting}{h}{lop}[chapter]}
\floatname{codelisting}{Listing}
\newcommand*\listoflistings{\listof{codelisting}{List of Listings}}
\makeatother
\makeatletter
\makeatother
\makeatletter
\@ifpackageloaded{caption}{}{\usepackage{caption}}
\@ifpackageloaded{subcaption}{}{\usepackage{subcaption}}
\makeatother
\ifLuaTeX
  \usepackage{selnolig}  % disable illegal ligatures
\fi
\usepackage{bookmark}

\IfFileExists{xurl.sty}{\usepackage{xurl}}{} % add URL line breaks if available
\urlstyle{same} % disable monospaced font for URLs
\hypersetup{
  pdftitle={From Text to Insight - A Novel Approach to Measuring Business Model Innovation},
  pdfauthor={Max Gabler; Wanshu Jiang; Christoph Kiesl; Leonard Pöhls; Alexander Rieber; Ansgar Scherp},
  pdfkeywords={10-K, Business Model Innovation, BERT, Gemini},
  colorlinks=true,
  linkcolor={blue},
  filecolor={Maroon},
  citecolor={Blue},
  urlcolor={Blue},
  pdfcreator={LaTeX via pandoc}}

\newcommand{\runninghead}{A Preprint }
\title{From Text to Insight - A Novel Approach to Measuring Business
Model Innovation}
\def\asep{\\\\\\ } % default: all authors on same column
\author{\textbf{Max
Gabler}\\\href{mailto:max.gabler@uni-ulm.de}{max.gabler@uni-ulm.de}\asep\textbf{Wanshu
Jiang}\\\href{mailto:wanshu.jiang@uni-ulm.de}{wanshu.jiang@uni-ulm.de}\asep\textbf{Christoph
Kiesl}\\\href{mailto:christoph.kiesl@uni-ulm.de}{christoph.kiesl@uni-ulm.de}\asep\textbf{Leonard
Pöhls}\\\href{mailto:leonard.poehls@uni-ulm.de}{leonard.poehls@uni-ulm.de}\asep\textbf{Alexander
Rieber}\\\href{mailto:alexander.rieber@uni-ulm.de}{alexander.rieber@uni-ulm.de}\asep\textbf{Ansgar
Scherp}\\\href{mailto:ansgar.scherp@uni-ulm.de}{ansgar.scherp@uni-ulm.de}}
\date{}
\begin{document}
\maketitle
\begin{abstract}
The ability of a company to continuously innovate its business model is
a pivotal determinant of long-term success in dynamic markets. It is
therefore crucial to ensure the reliability of business model innovation
measurement. In this study, we utilise business descriptions from 10-K
filings between 2017 and 2023 to measure business model innovation. The
methodology employed is as follows: Firstly, we employ Google's Gemini
to summarize the core elements of the business model from the 10-K
reports. Subsequently, we apply a BERT model (Bidirectional Encoder
Representation from Transformers) and calculate the similarity between
the summaries of a company over the observation period with the
BERTScore. This approach enables us to identify changes in a company's
business model over time. To evaluate the effectiveness of our measure,
we regress firm performance on the distance between two summaries in
consecutive years. Based on the existing literature, we hypothesise that
there will be a positive relationship. We find that (\ldots). The
findings of this study offer insights into the extent to which textual
similarities in regulatory reports can be employed as a reliable
indicator for business model innovation. Thus, this method represent a
novel approach to analyzing business model innovation over time.
\end{abstract}
{\bfseries \emph Keywords}
\def\sep{\textbullet\ }
10-K \sep Business Model Innovation \sep BERT \sep 
Gemini


\newpage{}

\section{Introduction}\label{introduction}

Business model innovation (BMI) is a key activity to maintain
competitiveness and even gain a competitive advantage (Pucihar et al.
2019; Teece 2018). It is therefore no surprise that the interest in BMI
has grown rapidly over the last twenty years. In particular, research
examining the impact of BMI on firm performance has been a prominent
area of investigation, with numerous research papers published in this
field (Cucculelli and Bettinelli 2015; Latifi, Nikou, and Bouwman 2021;
Zott and Amit 2008; White et al. 2022). While the financial literature
offers a wide range of established methods for measuring a company's
performance, the BMI literature provides only a limited number of
measures, all of which face similar challenges (White et al. 2022).
Furthermore, these measures vary largely. In order to further validate
and advance the BMI research field, more sophisticated and comprehensive
measurement instruments are necessary (Huang and Ichikohji 2023).

Scales and Measures used in the BMI literature (Clauss 2017; Spieth and
Schneider 2016) provide managers and practitioners with a measurement
index for business model innovativeness. But these measures only
validate applicability of BMI theory (Huang and Ichikohji 2023) and are
insufficient for longitudinal studies (Clauss 2017). Hence, these
measures are not adequate for a time series analysis of BMI.
Furthermore, they refer only to BMI as new-to-the-firm and are not able
to grasp BMI in the sense of new-to-the-industry and new-to-market. This
gap is addressed by proposing a novel approach to measuring BMI.
US-based companies are obliged by the United States Security and
Exchange Commission (SEC) to submit annual 10-K filings, wherein a
detailed description of the company's business operations is required.
Hoberg \& Phillips (2016), on which this study builds, use these filings
to create word vectors about the companies products in order to cluster
them into industries. We, on the other hand summarize, these
descriptions with Gemini, calculate the similarities between companies
and thereby cluster them into industries. Furthermore, we calculate the
BERTScore between the summaries of different years for a single company.
This approach enables the measurement of changes in the business model
(BM) over time as the distance between the BM summary of one year to
another. There is evidence that an increase in BMI is associated with
improved firm performance (Cucculelli and Bettinelli 2015; Latifi,
Nikou, and Bouwman 2021; White et al. 2022). In order to test the
validity of our measure, we regress revenue growth on our measure.

\begin{itemize}
\tightlist
\item
  Key findings (and Contribution, already down below)
\end{itemize}

Our contribution is made in two ways. Firstly, we build on the concept
of alternative industry classification put forth by Hoberg \& Phillips
(2016) and propose an industry classification system based on a firm's
BM. Secondly, we propose a novel measure for BMI that is sufficient for
longitudinal studies.

The SEC mandates that the majority of public companies based in the
United States submit specific documents in certain intervals. One such
document is the annual 10-K filing. These filings follow a set order of
topics and contain a range of information, including details about
managerial discussions, risk factors for the company, legal proceedings
and financial data. In the first section under the subtitle
``Business,'' a company presents its general business, encompassing
information about its products and services. In some instances
additional topics may be addressed, such as labor issues or competition
(SEC 2024). In conclusion, this section contains the most useful
information for describing a company's BM (Lee and Hong 2014).
Furthermore, 10-K filings are a reliable source of information, given
that US law prohibits false or misleading statements in the filings. The
SEC monitors the compliance of the companies with the requirements and
comments where disclosure appears to be inconsistent (SEC 2024).

\begin{itemize}
\tightlist
\item
  paragraph 5 (robustness checks)
\end{itemize}

In spite of the growing interest in BMI and the increasing number of
theoretical and empirical studies in this field, the research of BMI is
still in a preliminary state (Huang and Ichikohji 2023). Consequently,
there is considerable variation in the definitions of BMI, with some
definitions being more similar to one another than others (Foss and
Saebi 2017). Spieth \& Schneider (2016) identify three core dimensions a
company's BM is comprised of: its value proposition, its value creation
architecture and its revenue model logic. Based on this, BMI can be
conceptualized as a change that is new-to-the-firm in at least one of
these dimensions. Furthermore, Spieth and Schneider (2016) introduce a
measurement model to evaluate these three dimensions of BMI. They
develop an index by first specifying the contents, followed by a
specification of the indicators and assessing their content validity,
assessing the indicators collinearity and finally assessing the external
validity. A total of twelve indicators for measuring the innovativeness
of the BM were identified through a comprehensive literature review and
through engagement with industry practitioners. The external validity of
the formative indicators was successfully validated through a survey of
200 experts in strategy and innovation management (Spieth and Schneider
2016). Clauss (2017) employs a very similar approach. After specifying
the domain and dimensionality of BMI trough literature research, the
author divides his scale into three hierarchical levels consisting of 41
reflective items, 10 subconstructs and three main dimensions, which are
similar to the ones mentioned earlier. The scale was validated through
two samples from the manufacturing industry and further demonstrated
nomological validity (Clauss 2017). However, both measures are subject
to three significant limitations. Firstly, both measures lack a temporal
component. Consequently, they are inadequate for use in longitudinal
studies or ex-post evaluations of BMI. Secondly, BMI is only measured at
the new-to-the-firm level rather than at the new-to-the-industry or
new-to-the-market level. Thirdly, both measures rely on interviews and
questionnaires, which makes conducting large-scale studies
time-consuming and reliant on the willingness of the companies to
cooperate (Clauss 2017; Spieth and Schneider 2016).

The process of text mining 10-K filings is not a novel concept. Hoberg
\& Phillips (2016) present a novel approach to defining industry
boundaries. This is achieved through the parsing of the product
descriptions provided by firm 10-K filings and creating word vectors.
Specifically, the authors identify and exclude proper nouns, which
include common words and geographic locations. They then create word
vectors for each firm and year, which enables the measurement of product
similarity over time. They propose two novel industry classification
methods: the FIC and the text-based network industry classification
(TNIC). Firstly, they cluster companies based on the similarity of the
word vectors into fixed industries. Secondly, they define a minimum
similarity threshold, above which firms are considered in the same
industry. This relaxes their prior properties of binary membership
transitivity and fixed industry location. This way the authors
demonstrate shortcomings in the traditional industry classification
systems such as the Standard Industry Classification (SIC) and the North
American Industry Classification System (NAICS), which are not able to
account for temporal changes. The new method is capable of capturing
changes in industry boundaries and competitor sets over time, thereby
providing a dynamic industry classification system. In their study, Lee
\& Hong (2014) examine the evolution of a firm's BM over time. The
authors represent each document as a vector of keywords, which is
similar to the approach utilized by Hoberg \& Phillips (2016). After
identifying the Item 1 part of the 10-K filings as the most crucial part
for describing a firm's BM, Lee \& Hong (2014) filter these for relevant
sentences. Subsequently, the authors construct keyword vectors, which
represent the concept of the BM. Therefore, the evolution of the BM is
depicted as the change in the distribution of keywords over time.
Nevertheless, this approach is not without shortcomings. The authors
advocate for a more robust methodology, such as incorporating multi-word
phrases in the keyword vectors, to enhance the reliability of the
approach (Lee and Hong 2014).

The rest of the paper proceeds as follows. Section 2 describes the
preprocessing with Gemini, our data and our methodology. Section 3 our
estimations strategy. Section 4 provides a comparison of the BERTScore
Classification and the FIC as well as the discussion of our results.
Section 5 concludes our study.

\newpage{}

\section{Data and Methodology}\label{data-and-methodology}

\subsection{Preprocessing with Gemini}\label{preprocessing-with-gemini}

10-K filings are typically very large text documents, and Item 1 of
these filings is no exception. Table 1 shows the descriptive measures of
the length of the original Item 1 section in our final sample. The
length of a document was measured by the word count without punctuation.
The document length ranges from a couple hundred words to tens of
thousands. In order to utilise the entirety of the information regarding
the BM in the Item 1 section and pass the text to our BERT model, we
decided to let Google's GenAI chatbot Gemini summarize them to a maximum
length of 512 tokens. The summaries were created between 26 June 2024
and 6 August 2024. The model employed was Gemini Flash 1.5. The prompt
was inserted at the beginning of each text file and it was passed via an
API to Gemini \footnote{We forked and used following Github repository:
  https://github.com/skranz/gemini\_ex.}. We used following prompt:
``Summarize the business model from the following text. Answer with a
continuous text and with five hundred twelve tokens at max. Set your
focus on sources of revenue, the intended customer base, products,
distribution channels and details of financing. Use only information
from the following the text''.\footnote{The spelling error in the last
  sentence of the prompt was found after processing Item 1. After
  evaluating the summaries, this error did not cause any issues.}
``intended customer base'' and ``product'' refer to the value offering,
``distribution channels'' refers to the value architecture, and
``sources of revenue'' and ``details of financing'' refer to the revenue
model. Consequently, this prompt covers all aspects of the definition of
BMI proposed by Spieth \& Schneider (2016). The term `tokens' was used
deliberately in preference to `words', given that the number of tokens
and the number of words in a text may vary depending on the tokeniser.
This way, we wanted to ensure that the whole summary is used by the BERT
model. To assess the quality and accuracy of the summaries produced by
Gemini, a random sample of 100 filings was selected for comparison with
the original text. More precise,the original file was initially read
with a focus on the points mentioned in the prompt. Subsequently, the
summary was evaluated to ascertain whether it contained these same
points. A list of the sample with the summaries is provided in the
Appendix.

\begin{itemize}
\tightlist
\item
  result of this check
\end{itemize}

\begin{figure}

\begin{minipage}{\linewidth}

\begin{longtable}[]{@{}
  >{\centering\arraybackslash}p{(\columnwidth - 14\tabcolsep) * \real{0.0645}}
  >{\centering\arraybackslash}p{(\columnwidth - 14\tabcolsep) * \real{0.0753}}
  >{\centering\arraybackslash}p{(\columnwidth - 14\tabcolsep) * \real{0.2151}}
  >{\centering\arraybackslash}p{(\columnwidth - 14\tabcolsep) * \real{0.0968}}
  >{\centering\arraybackslash}p{(\columnwidth - 14\tabcolsep) * \real{0.1828}}
  >{\centering\arraybackslash}p{(\columnwidth - 14\tabcolsep) * \real{0.0860}}
  >{\centering\arraybackslash}p{(\columnwidth - 14\tabcolsep) * \real{0.1828}}
  >{\centering\arraybackslash}p{(\columnwidth - 14\tabcolsep) * \real{0.0968}}@{}}
\caption{Descriptive Statistics Original Filings}\tabularnewline
\toprule\noalign{}
\begin{minipage}[b]{\linewidth}\centering
Year
\end{minipage} & \begin{minipage}[b]{\linewidth}\centering
Mean
\end{minipage} & \begin{minipage}[b]{\linewidth}\centering
Standard Deviation
\end{minipage} & \begin{minipage}[b]{\linewidth}\centering
Minimum
\end{minipage} & \begin{minipage}[b]{\linewidth}\centering
25th Percentile
\end{minipage} & \begin{minipage}[b]{\linewidth}\centering
Median
\end{minipage} & \begin{minipage}[b]{\linewidth}\centering
75th Percentile
\end{minipage} & \begin{minipage}[b]{\linewidth}\centering
Maximum
\end{minipage} \\
\midrule\noalign{}
\endfirsthead
\toprule\noalign{}
\begin{minipage}[b]{\linewidth}\centering
Year
\end{minipage} & \begin{minipage}[b]{\linewidth}\centering
Mean
\end{minipage} & \begin{minipage}[b]{\linewidth}\centering
Standard Deviation
\end{minipage} & \begin{minipage}[b]{\linewidth}\centering
Minimum
\end{minipage} & \begin{minipage}[b]{\linewidth}\centering
25th Percentile
\end{minipage} & \begin{minipage}[b]{\linewidth}\centering
Median
\end{minipage} & \begin{minipage}[b]{\linewidth}\centering
75th Percentile
\end{minipage} & \begin{minipage}[b]{\linewidth}\centering
Maximum
\end{minipage} \\
\midrule\noalign{}
\endhead
\bottomrule\noalign{}
\endlastfoot
2016 & 7842 & 6104 & 155 & 3705 & 6026 & 10271 & 51227 \\
2017 & 7542 & 6320 & 155 & 3522 & 5767 & 9700 & 70611 \\
2018 & 7604 & 6272 & 180 & 3528 & 5771 & 9669 & 71700 \\
2019 & 8009 & 6631 & 189 & 3669 & 5971 & 10410 & 78270 \\
2020 & 8660 & 7195 & 171 & 3943 & 6449 & 10971 & 57980 \\
2021 & 10324 & 8406 & 235 & 4670 & 7568 & 13563 & 78799 \\
2022 & 9471 & 7997 & 171 & 4309 & 7042 & 11897 & 73937 \\
2023 & 6646 & 4771 & 190 & 3660 & 5814 & 8401 & 43523 \\
\end{longtable}

\end{minipage}%

\end{figure}%

\subsection{The Dataset}\label{the-dataset}

We collect 10-K filings from the digital SEC Database, using the
category ``10-K'' as extraction condition. Since the focus of our study
lies on company's BM, we merely use the Item 1 part, since this is the
most crucial part of the 10-K filings for describing the companies BM
(Lee and Hong 2014). Our observations are limited to an intersection of
such companies, which on the one hand has been made available to the SEC
since 2001 in a publicly accessible list of 10,284 companies (Appendix),
of which 7,590 are currently listed on NASDAQ, NYSE or over-the-counter.
We extracted 10-K filings that were submitted between 2017 and 2023
based on underlying Central Index Keys (CIK). We exclude companies from
the financial sector, namely companies with a SIC Code starting with
six. Corresponding to Table 2, multiple steps of pre-processing were
required to obtain the final amount of 21,683 observations for seven
years. Financial key figures, including net income, total assets and
others were originally extracted from the SEC, but also challenged with
financial values from DataStream. A total of 4,225 companies are
included in the sample, although the availability of filings could not
always be guaranteed for all years. This is due on the one hand to the
quality of the API to the SEC and on the other hand to companies that
did not file 10-K reports or were listed on the stock exchange for the
entire period under review. Finally, we have access to the financial key
figures of the companies for the respective year, the Item I text
pre-processed with the help of Gemini, company-specific identification
features and the conventional SIC industry classification.

\begin{figure}

\begin{minipage}{\linewidth}

\begin{longtable}[t]{>{\centering\arraybackslash}p{8cm}cc}
\caption{10-K Sample Creation}\tabularnewline

\toprule
Source/Filter & Sample Size & Observations Removed\\
\midrule
1. Extracted 10-K filings from the SEC & 35731 & \\
2. Excluding filings from financial companies (SIC-code starting with '6') & 27569 & 8162\\
3. Verify for Item 1 text availability (removed oberservations that are attributable to API quality) & 23982 & 3587\\
4. Extracting dates for which the filings are reporting for and removing of duplicated filings & 23971 & 11\\
5. Delete observations with incorrect date assignment & 22161 & 1810\\
\addlinespace
6. Merged Gemini processed Item 1 text to the underlying data set. We did not consider texts that were not processable & 21697 & 464\\
7. Extract financials statements from SEC and merge them. Also remove observations for years prior to 2016 & 21686 & 11\\
\bottomrule
\multicolumn{3}{l}{\rule{0pt}{1em}\textit{Note: }}\\
\multicolumn{3}{l}{\rule{0pt}{1em}Filings submitted between 2017 and 2023 are considered}\\
\end{longtable}

\end{minipage}%

\end{figure}%

TODO

\begin{itemize}
\tightlist
\item
  Descriptive Table3 for length of summary
\item
  Description of Table2, Table3, and the final dataset
\end{itemize}

\subsection{BERT and BERTScore}\label{bert-and-bertscore}

BERT is a pre-trained and transformer-based model for natural language
processing (NPL) based on artificial neural networks. It works according
to the so-called transformer architecture, which was first mentioned by
Vaswani et al. (2017). According to these authors, this architecture
consists of two main components, the encoder and the decoder. The
encoder consists of several identical layers, which initially use the
so-called self-attention mechanism to generate context-dependent
representations of each word in the sentence. This mechanism can be
parallelized and therefore enables different aspects of the context to
be captured in the same way. The decoder, on the other hand, works in a
similar way and is responsible for processing the information from the
encoder and forming it into an output sequence. However, this is not
relevant for BERT, as no sequence-to-sequence transformation is carried
out in BERT. In contrast to Hoberg \& Philips (2016) word-to-vec
approach, BERT works bidirectional and takes into account the context
from both sides of each word simultaneously. Therefore, BERT is able to
capture deeper semantics in texts such as 10-K reports. The BERTScore
now computes the cosine similarity between word or text meanings, that
have been determined by representations (or embeddings) learned from
BERT. The scale is from -1 to 1, where 1 describes a perfect similarity.

\subsection{Methodology}\label{methodology}

After processing the data, we calculate the BERTScore between the
summaries of different companies in one year and between the summaries
for the same company over different years. The similarity between
different companies in the same year is utilized to compare our
BERTScore industry classification with the FIC by Hoberg \& Phillips
(2016) and the SIC. Hoberg \& Phillips (2016) calculate the cosine
similarity between word vectors of product descriptions while we utilise
the BERTScore to calculate the similarity between our BM summaries. The
methodology and object of research differ between the two studys. But
the product of a company is by the definition of Spieth \& Schneider
(2016) a part of its value offering and thereby a part of the BM.
Because the product is thereby entangled with the BM, companies that
have similar products might have similar BMs. So despite the different
methodology and object of research, we expect a similar distribution as
Hoberg for the FIC, which is very granular and contains lot of small
industries. Thus, we hypothesize:

\textbf{H1}: Our industry classification shows a similar distribution
compared to the FIC.

\textbf{H2}: Our industry classification has a high overlap with the
FIC.

As mentioned, our approach differs not marginally from the original
paper by Hoberg \& Phillips (2016). We fix the company and calculate the
BERTScore between the summaries of different years. These summaries
describe the BM of a company based on the 10-K filing of that company
for that year. When a company innovates its BM over time, the 10-K
filings change and thus the summaries of these filings. By taking the
similarity between this summaries, we are able to calculate the distance
between the summaries. This allows us to measure the change in a
company's BM. Under the assumption that companies ``don't run in
circles'', this measures BMI on the new-to-the-firm level. The
assumption ``don't run in circles'' means, that a company that not
marginally changes its BM, won't change it back to the same BM it had
before. In the BMI literature, the positive relationship between BMI and
firm performance is well explored by a wealth of studies examining this
relationship (White et al. 2022). In the case that our measure indeed
measures BMI, we expect to find a positive relationship between our BMI
measure and firm performance. Therefore, we hypothesize that:

\textbf{H3}: Our measure for BMI shows a positive relationship with firm
performance.

\section{Estimation Strategy}\label{estimation-strategy}

A number of studies have examined the relationship between BMI and the
financial performance of a company. Cucculelli \& Bettinelli (2015)
investigate the effect of BMI on sales growth, return on sales (ROS) and
total factor productivity (TFP). The results provide support for the
hypothesis that BMI has a positive effect on firm performance, with the
effect increasing in line with the intensity of the innovation. Latifi
et al. (2021) also find evidence that provides support for the
hypothesis that a company engaging in BMI will improve its overall firm
performance. In their study, the firm performance is evaluated
subjectively. Zott \& Amit (2008) analyse the effect of the BM and the
product market strategy on firm performance. They measure the firm
performance with the market value of equity as the stock price
multiplied with the number of shares outstanding. White et al. (2022)
conducted a meta-analysis based on the extant BMI literature. They found
a positive relationship between BMI and firm performance, and that this
relationship is shaped by factors including the firm age, industry, the
economic and political environment and BMI characteristics.

Following paragraph is still preliminary: - Based on the literature
(mostly on (White et al. 2022) and (Zott and Amit 2008)) we build our
estimation strategy; use multiple multivariate regressions; - dependent
variables: revenue growth, market equity (growth), Tobins Q -
independent variable: distance between summaries (makes the measure more
intuitive; if distance grows/ similarity decreases, summaries are more
dissimilar -\textgreater{} BMI) - controls: firm age and size, might
need to consider how to measure size (employees, firm value), Tobins Q
might also be used - fixed effects for year and industry might me
useful.

\section{Results and Discussion}\label{results-and-discussion}

\subsection{Comparison}\label{comparison}

Our study builds on the idea of Hoberg \& Phillips (2016) to utilize
text data from 10-K filings to classify companies based on their
similarity to each other into dynamic industries. Our approach differs
in two ways: Firstly, in contrast to the TNIC and FIC, which employ
word-to-vec, our approach utilises BERT to represent text. Accordingly,
the BERTScore is employed instead of the cosine similarity as our
similarity measure. Secondly, our analysis is focused on the description
of the BM rather than on the product descriptions. Nevertheless, in the
following subsection, the BERTScore industry classification is compared
with the FIC and the SIC.

The data employed for the industry classification with the BERTScore is
the same as described in Section 2. The SIC codes come from the SEC
website\footnote{The list can be found here:
  https://www.sec.gov/search-filings/standard-industrial-classification-sic-code-list.}.
For the FIC we have utilized the similarity scores provided by
Hoberg-Phillips Data Library.\footnote{For the database see:
  https://hobergphillips.tuck.dartmouth.edu.} The data consists of the
gvkeys of two companies, the year and the cosine similarity between
these two companies. In order to ensure comparability, only companies
present in both the present study's dataset and that provided by the
authors are included in the analysis. Because we use CIKs and accession
numbers to identify firms and filings, and the fact that the data
library employs Compustat's gvkeys, the matching of CIKs with gvkeys
inevitably results in the loss of some observations. Ultimately, the
clustering algorithm was applied to 1,958 firms for the year 2017.
Hoberg \& Phillips (2016) perform two steps to create the FIC. Firstly,
a hierarchical agglomerative clustering algorithm is employed to cluster
companies based on their similarity and maximize ex-post within cluster
similarity. This enables a classification with any number of clusters.
In our dataset, companies are from 320 different SIC codes. Therefore,
the number of industries chosen for our industry classification and the
FIC is 320. In the second step, the authors compute aggregated word
vectors for each industry. These vectors now represent the industries.
Subsequently, the similarity between industries and firms is calculated
for each of the following years. From the second year onwards, firms are
classified according to the industry with which they are most similar.
But due to our methodology this step is omitted.

Figure 1 presents a comparison of the distribution of industry size for
the BERTScore classification, the FIC and the SIC. Both the BERTScore
classification and the FIC show a similar distribution, displaying a
leftward skew with the majority of industries comprising fewer than ten
firms. The SIC shows as well a left skewed distribution but with most
industries only containing one company. The distribution of the FIC is
steeper than the on of the BERTScore classification. It is notable that
the largest industry in the BERTScore classification comprises only 20
companies, whereas the FIC and SIC contain industries with a greater
number of firms, with some exceeding 50. This suggests that the
BERTScore classification groups small to medium-sized industries,
comprising between two and fourteen firms per industry, with fewer large
industries. The FIC also comprises mostly of small to medium-sized
industries, with a few larger ones. Despite these minor differences,
this supports H1. The degree of homogeneity between the BERTScore
classification and the FIC is 0.63, while the completeness is 0.6. This
demonstrates only a medium degree of overlap between the two
classifications. The Adjusted Rand Index (ARI) (Hubert and Arabie 1985)
is situated at 0.0002, which is close to zero, indicating that the
overlap is random. These findings do not provide support for H2.

\begin{itemize}
\tightlist
\item
  explain how we further utilize our classification in the estimation
\end{itemize}

\includegraphics{ProjectEcoDataScience_files/figure-pdf/unnamed-chunk-4-1.pdf}

\subsection{Results}\label{results}

\subsection{Robustness Checks}\label{robustness-checks}

\subsection{Discussion}\label{discussion}

\section{Conclusion}\label{conclusion}

\section{Acknowledgement}\label{acknowledgement}

\begin{itemize}
\tightlist
\item
  Jonathan for IT Support
\item
  Prof.~Kranz for Repo
\end{itemize}

\newpage{}

\section{References}\label{references}

\phantomsection\label{refs}
\begin{CSLReferences}{1}{0}
\bibitem[\citeproctext]{ref-clauss_measuring_2017}
Clauss, Thomas. 2017. {``Measuring Business Model Innovation:
Conceptualization, Scale Development, and Proof of Performance.''}
\emph{R\&D Management} 47 (3): 385--403.
\url{https://doi.org/10.1111/radm.12186}.

\bibitem[\citeproctext]{ref-cucculelli_business_2015}
Cucculelli, Marco, and Cristina Bettinelli. 2015. {``Business Models,
Intangibles and Firm Performance: Evidence on Corporate Entrepreneurship
from {Italian} Manufacturing {SMEs}.''} \emph{Small Business Economics}
45 (2): 329--50. \url{https://doi.org/10.1007/s11187-015-9631-7}.

\bibitem[\citeproctext]{ref-foss_fifteen_2017}
Foss, Nicolai J., and Tina Saebi. 2017. {``Fifteen {Years} of {Research}
on {Business} {Model} {Innovation}: {How} {Far} {Have} {We} {Come}, and
{Where} {Should} {We} {Go}?''} \emph{Journal of Management} 43 (1):
200--227. \url{https://doi.org/10.1177/0149206316675927}.

\bibitem[\citeproctext]{ref-hoberg_text-based_2016}
Hoberg, Gerard, and Gordon Phillips. 2016. {``Text-{Based} {Network}
{Industries} and {Endogenous} {Product} {Differentiation}.''}
\emph{Journal of Political Economy} 124 (5): 1423--65.
\url{https://doi.org/10.1086/688176}.

\bibitem[\citeproctext]{ref-huang_review_2023}
Huang, WenJun, and Takeyasu Ichikohji. 2023. {``A Review and Analysis of
the Business Model Innovation Literature.''} \emph{Heliyon} 9 (7):
e17895. \url{https://doi.org/10.1016/j.heliyon.2023.e17895}.

\bibitem[\citeproctext]{ref-hubert_comparing_1985}
Hubert, Lawrence, and Phipps Arabie. 1985. {``Comparing Partitions.''}
\emph{Journal of Classification} 2 (1): 193--218.
\url{https://doi.org/10.1007/BF01908075}.

\bibitem[\citeproctext]{ref-latifi_business_2021}
Latifi, Mohammad-Ali, Shahrokh Nikou, and Harry Bouwman. 2021.
{``Business Model Innovation and Firm Performance: {Exploring} Causal
Mechanisms in {SMEs}.''} \emph{Technovation} 107 (September): 102274.
\url{https://doi.org/10.1016/j.technovation.2021.102274}.

\bibitem[\citeproctext]{ref-lee_business_2014}
Lee, Jihwan, and Yoo S. Hong. 2014. {``Business {Model} {Mining}:
{Analyzing} a {Firm}'s {Business} {Model} with {Text} {Mining} of
{Annual} {Report}.''} \emph{Industrial Engineering and Management
Systems} 13 (4): 432--41.
\url{https://doi.org/10.7232/iems.2014.13.4.432}.

\bibitem[\citeproctext]{ref-pucihar_drivers_2019}
Pucihar, Andreja, Gregor Lenart, Mirjana Kljajić Borštnar, Doroteja
Vidmar, and Marjeta Marolt. 2019. {``Drivers and {Outcomes} of
{Business} {Model} {Innovation}---{Micro}, {Small} and {Medium}-{Sized}
{Enterprises} {Perspective}.''} \emph{Sustainability} 11 (2): 344.
\url{https://doi.org/10.3390/su11020344}.

\bibitem[\citeproctext]{ref-sec_investor_2024}
SEC. 2024. {``Investor {Bulletin}: {How} to {Read} a 10-{K}.''}
\url{https://www.sec.gov/files/reada10k.pdf}.

\bibitem[\citeproctext]{ref-spieth_business_2016}
Spieth, Patrick, and Sabrina Schneider. 2016. {``Business Model
Innovativeness: Designing a Formative Measure for Business Model
Innovation.''} \emph{Journal of Business Economics} 86 (6): 671--96.
\url{https://doi.org/10.1007/s11573-015-0794-0}.

\bibitem[\citeproctext]{ref-teece_business_2018}
Teece, David J. 2018. {``Business Models and Dynamic Capabilities.''}
\emph{Long Range Planning} 51 (1): 40--49.
\url{https://doi.org/10.1016/j.lrp.2017.06.007}.

\bibitem[\citeproctext]{ref-vaswani2017attention}
Vaswani, Ashish, Noam Shazeer, Niki Parmar, Jakob Uszkoreit, Llion
Jones, Aidan N Gomez, Lukasz Kaiser, and Illia Polosukhin. 2017.
{``Attention Is All You Need.(nips), 2017.''} \emph{arXiv Preprint
arXiv:1706.03762} 10: S0140525X16001837.

\bibitem[\citeproctext]{ref-white_exploring_2022}
White, Joshua V., Erik Markin, David Marshall, and Vishal K. Gupta.
2022. {``Exploring the Boundaries of Business Model Innovation and Firm
Performance: {A} Meta-Analysis.''} \emph{Long Range Planning} 55 (5):
102242. \url{https://doi.org/10.1016/j.lrp.2022.102242}.

\bibitem[\citeproctext]{ref-zott_fit_2008}
Zott, Christoph, and Raphael Amit. 2008. {``The Fit Between Product
Market Strategy and Business Model: Implications for Firm
Performance.''} \emph{Strategic Management Journal} 29 (1): 1--26.
\url{https://doi.org/10.1002/smj.642}.

\end{CSLReferences}

\newpage{}

\section{Appendix}\label{appendix}

\subsection{Appendix A}\label{appendix-a}

\begin{itemize}
\tightlist
\item
  lore ipsum
\end{itemize}

\includegraphics{ProjectEcoDataScience_files/figure-pdf/unnamed-chunk-6-1.pdf}

\subsection{Appendix B}\label{appendix-b}

\begin{figure}%
\end{figure}%



\end{document}
