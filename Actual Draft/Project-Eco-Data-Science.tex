\documentclass[manuscript,screen,review]{acmart}


\IfFileExists{upquote.sty}{\usepackage{upquote}}{}
\IfFileExists{microtype.sty}{% use microtype if available
  \usepackage[]{microtype}
  \UseMicrotypeSet[protrusion]{basicmath} % disable protrusion for tt fonts
}{}
\makeatletter
\@ifundefined{KOMAClassName}{% if non-KOMA class
  \IfFileExists{parskip.sty}{%
    \usepackage{parskip}
  }{% else
    \setlength{\parindent}{0pt}
    \setlength{\parskip}{6pt plus 2pt minus 1pt}}
}{% if KOMA class
  \KOMAoptions{parskip=half}}
\makeatother

%%
%% This is file `sample-manuscript.tex',
%% generated with the docstrip utility.
%%
%% The original source files were:
%%
%% samples.dtx  (with options: `manuscript')
%% 
%% IMPORTANT NOTICE:
%% 
%% For the copyright see the source file.
%% 
%% Any modified versions of this file must be renamed
%% with new filenames distinct from sample-manuscript.tex.
%% 
%% For distribution of the original source see the terms
%% for copying and modification in the file samples.dtx.
%% 
%% This generated file may be distributed as long as the
%% original source files, as listed above, are part of the
%% same distribution. (The sources need not necessarily be
%% in the same archive or directory.)
%%
%%
%% Commands for TeXCount
%TC:macro \cite [option:text,text]
%TC:macro \citep [option:text,text]
%TC:macro \citet [option:text,text]
%TC:envir table 0 1
%TC:envir table* 0 1
%TC:envir tabular [ignore] word
%TC:envir displaymath 0 word
%TC:envir math 0 word
%TC:envir comment 0 0
%%
%%
%% The first command in your LaTeX source must be the \documentclass command.


% Options for packages loaded elsewhere
\PassOptionsToPackage{unicode}{hyperref}
\PassOptionsToPackage{hyphens}{url}
\PassOptionsToPackage{dvipsnames,svgnames,x11names}{xcolor}

\IfFileExists{bookmark.sty}{\usepackage{bookmark}}{\usepackage{hyperref}}

%% PANDOC PREAMBLE BEGINS


\providecommand{\tightlist}{%
  \setlength{\itemsep}{0pt}\setlength{\parskip}{0pt}}\usepackage{longtable,booktabs,array}
\usepackage{calc} % for calculating minipage widths
% Correct order of tables after \paragraph or \subparagraph
\usepackage{etoolbox}
\makeatletter
\patchcmd\longtable{\par}{\if@noskipsec\mbox{}\fi\par}{}{}
\makeatother
% Allow footnotes in longtable head/foot
\IfFileExists{footnotehyper.sty}{\usepackage{footnotehyper}}{\usepackage{footnote}}
\makesavenoteenv{longtable}
\usepackage{graphicx}
\makeatletter
\def\maxwidth{\ifdim\Gin@nat@width>\linewidth\linewidth\else\Gin@nat@width\fi}
\def\maxheight{\ifdim\Gin@nat@height>\textheight\textheight\else\Gin@nat@height\fi}
\makeatother
% Scale images if necessary, so that they will not overflow the page
% margins by default, and it is still possible to overwrite the defaults
% using explicit options in \includegraphics[width, height, ...]{}
\setkeys{Gin}{width=\maxwidth,height=\maxheight,keepaspectratio}
% Set default figure placement to htbp
\makeatletter
\def\fps@figure{htbp}
\makeatother

\definecolor{mypink}{RGB}{219, 48, 122}
\makeatletter
\@ifpackageloaded{caption}{}{\usepackage{caption}}
\AtBeginDocument{%
\ifdefined\contentsname
  \renewcommand*\contentsname{Table of contents}
\else
  \newcommand\contentsname{Table of contents}
\fi
\ifdefined\listfigurename
  \renewcommand*\listfigurename{List of Figures}
\else
  \newcommand\listfigurename{List of Figures}
\fi
\ifdefined\listtablename
  \renewcommand*\listtablename{List of Tables}
\else
  \newcommand\listtablename{List of Tables}
\fi
\ifdefined\figurename
  \renewcommand*\figurename{Figure}
\else
  \newcommand\figurename{Figure}
\fi
\ifdefined\tablename
  \renewcommand*\tablename{Table}
\else
  \newcommand\tablename{Table}
\fi
}
\@ifpackageloaded{float}{}{\usepackage{float}}
\floatstyle{ruled}
\@ifundefined{c@chapter}{\newfloat{codelisting}{h}{lop}}{\newfloat{codelisting}{h}{lop}[chapter]}
\floatname{codelisting}{Listing}
\newcommand*\listoflistings{\listof{codelisting}{List of Listings}}
\makeatother
\makeatletter
\makeatother
\makeatletter
\@ifpackageloaded{caption}{}{\usepackage{caption}}
\@ifpackageloaded{subcaption}{}{\usepackage{subcaption}}
\makeatother
%% PANDOC PREAMBLE ENDS

\setlength{\parindent}{10pt}
\setlength{\parskip}{0pt}

\hypersetup{
  pdftitle={The Name of the Title Is Hope},
  pdfauthor={Max Gabler; Wanshu Jiang; Christoph Kiesl; Leonard Pöhls; Alexander Rieber; Ansgar Scherp},
  colorlinks=true,
  linkcolor={blue},
  filecolor={Maroon},
  citecolor={Blue},
  urlcolor={red},
  pdfcreator={LaTeX via pandoc, via quarto}}

%% \BibTeX command to typeset BibTeX logo in the docs
\AtBeginDocument{%
  \providecommand\BibTeX{{%
    Bib\TeX}}}

%% Rights management information.  This information is sent to you
%% when you complete the rights form.  These commands have SAMPLE
%% values in them; it is your responsibility as an author to replace
%% the commands and values with those provided to you when you
%% complete the rights form.
\setcopyright{acmcopyright}
\copyrightyear{2018}
\acmYear{2018}
\acmDOI{XXXXXXX.XXXXXXX}

%% These commands are for a PROCEEDINGS abstract or paper.
\acmConference[Conference acronym 'XX]{Make sure to enter the correct
conference title from your rights confirmation emai}{June 03--05,
2018}{Woodstock, NY}
\acmPrice{15.00}
\acmISBN{978-1-4503-XXXX-X/18/06}

%% Submission ID.
%% Use this when submitting an article to a sponsored event. You'll
%% receive a unique submission ID from the organizers
%% of the event, and this ID should be used as the parameter to this command.
%%\acmSubmissionID{123-A56-BU3}

%%
%% For managing citations, it is recommended to use bibliography
%% files in BibTeX format.
%%
%% You can then either use BibTeX with the ACM-Reference-Format style,
%% or BibLaTeX with the acmnumeric or acmauthoryear sytles, that include
%% support for advanced citation of software artefact from the
%% biblatex-software package, also separately available on CTAN.
%%
%% Look at the sample-*-biblatex.tex files for templates showcasing
%% the biblatex styles.
%%

%%
%% The majority of ACM publications use numbered citations and
%% references.  The command \citestyle{authoryear} switches to the
%% "author year" style.
%%
%% If you are preparing content for an event
%% sponsored by ACM SIGGRAPH, you must use the "author year" style of
%% citations and references.
%% Uncommenting
%% the next command will enable that style.
%%\citestyle{acmauthoryear}


%% end of the preamble, start of the body of the document source.
\begin{document}


%%
%% The "title" command has an optional parameter,
%% allowing the author to define a "short title" to be used in page headers.
\title[Hope]{The Name of the Title Is Hope}

%%
%% The "author" command and its associated commands are used to define
%% the authors and their affiliations.
%% Of note is the shared affiliation of the first two authors, and the
%% "authornote" and "authornotemark" commands
%% used to denote shared contribution to the research.


  \author{Max Gabler}
  
    \author{Wanshu Jiang}
  
    \author{Christoph Kiesl}
  
    \author{Leonard Pöhls}
  
    \author{Alexander Rieber}
  
    \author{Ansgar Scherp}
  
  
\renewcommand{\shortauthors}{Trovato et al.}

%% By default, the full list of authors will be used in the page
%% headers. Often, this list is too long, and will overlap
%% other information printed in the page headers. This command allows
%% the author to define a more concise list
%% of authors' names for this purpose.
%\renewcommand{\shortauthors}{Trovato et al.}
%%  
%% The abstract is a short summary of the work to be presented in the
%% article.
\begin{abstract}
A clear and well-documented \LaTeX~document is presented as an article
formatted for publication by ACM in a conference proceedings or journal
publication. Based on the ``acmart'' document class, this article
presents and explains many of the common variations, as well as many of
the formatting elements an author may use in the preparation of the
documentation of their work.    
\end{abstract}

%%
%% The code below is generated by the tool at http://dl.acm.org/ccs.cfm.
%% Please copy and paste the code instead of the example below.
%%
\begin{CCSXML}
<ccs2012>
 <concept>
  <concept_id>10010520.10010553.10010562</concept_id>
  <concept_desc>Computer systems organization~Embedded systems</concept_desc>
  <concept_significance>500</concept_significance>
 </concept>
 <concept>
  <concept_id>10010520.10010575.10010755</concept_id>
  <concept_desc>Computer systems organization~Redundancy</concept_desc>
  <concept_significance>300</concept_significance>
 </concept>
 <concept>
  <concept_id>10010520.10010553.10010554</concept_id>
  <concept_desc>Computer systems organization~Robotics</concept_desc>
  <concept_significance>100</concept_significance>
 </concept>
 <concept>
  <concept_id>10003033.10003083.10003095</concept_id>
  <concept_desc>Networks~Network reliability</concept_desc>
  <concept_significance>100</concept_significance>
 </concept>
</ccs2012>
\end{CCSXML}

\ccsdesc[500]{Computer systems organization~Embedded systems}
\ccsdesc[300]{Computer systems organization~Redundancy}
\ccsdesc{Computer systems organization~Robotics}
\ccsdesc[100]{Networks~Network reliability}

%%
%% Keywords. The author(s) should pick words that accurately describe
%% the work being presented. Separate the keywords with commas.
\keywords{10-K, Business Model Innovation, BERT, Gemini}


%%
%% This command processes the author and affiliation and title
%% information and builds the first part of the formatted document.
\maketitle

\setlength{\parskip}{-0.1pt}

\section{Introduction}\label{introduction}

Business model innovation (BMI) is a key activity to maintain
competitiveness and even gain a competitive advantage
\citep{pucihar_drivers_2019, teece_business_2018}. It is therefore no
surprise that the interest in BMI and methods of measuring it has grown
rapidly over the last twenty years. Researchers have recently called for
a BMI measurement instrument that is more comprehensive and advanced
than already existing ones \citep{huang_review_2023}. The scale
developed by Spieth \& Schneider \citeyearpar{spieth_business_2016}
provides managers and practitioners with a measurement index for
business model innovativeness. This measurement model only validates
applicability of BMI theory \citep{huang_review_2023} and is
insufficient for longitudinal studies \citep{clauss_measuring_2017}.
Hence, this measure is not adequate for a time series analysis of BMI.
Furthermore, it refers only to BMI as new-to-the-firm and is not able to
grasp BMI in the sense of new-to-the-industry and new-to-market.

\begin{itemize}
\item
  Solution approach (both after we actually are final with approach and
  have (some) results)
\item
  Results
\end{itemize}

Our contribution is made in a number of ways. Firstly, we tackle two
issues raised in the study by Lee \& Hong
\citeyearpar{lee_business_2014}: we employ a more reliable and
contemporary methodology for extracting the business model (BM) from
10-K filings and we are able to extend the scope of their study.
Secondly, we build on the concept of alternative industry classification
put forth by Hoberg \& Phillips \citeyearpar{hoberg_text-based_2016} and
propose an industry classification system based on a firm's BM. Thirdly,
we propose a novel measure for BMI that is sufficient for longitudinal
studies.

\section{Related Work}\label{related-work}

Our study borders several research fields and thus we have a variaty of
literature to our disposal. Section 2.1 presents state of the art
measures of BMI while Section 2.2 examines the relationship between BMI
and the financial performance of a firm. Section 2.3 deals with previous
approaches to text mine 10-K filings. Finally, in Section 2.4 a summary
of the findings from the related work is presented.

\subsection{Measuring Business Model
Innovation}\label{measuring-business-model-innovation}

In spite of the growing interest in BMI and the increasing number of
theoretical and empirical studies in this field, the research of BMI is
still in a preliminary state \citep{huang_review_2023}. Consequently,
there is considerable variation in the definitions of BMI, with some
definitions being more similar to one another than others
\citep{foss_fifteen_2017}. Spieth \& Schneider
\citeyearpar{spieth_business_2016} identify three core dimensions a
company's BM is comprised of: its value proposition, its value creation
architecture and its revenue model logic. Based on this, BMI can be
conceptualized as a change that is new-to-the-firm in at least one of
these dimensions. Furthermore, Spieth and Schneider
\citeyearpar{spieth_business_2016} introduce a measurement model to
evaluate these three dimensions of BMI. They develop an index by first
specifying the contents, followed by a specification of the indicators
and assessing their content validity, assessing the indicators
collinearity and finally assessing the external validity. A total of
twelve indicators for measuring the innovativeness of the BM were
identified through a comprehensive literature review and through
engagement with industry practitioners. The external validity of the
formative indicators was successfully validated through a survey of 200
experts in strategy and innovation management
\citep{spieth_business_2016}. Clauss \citeyearpar{clauss_measuring_2017}
employs a very similar approach. After specifying the domain and
dimensionality of BMI trough literature research, the author divides his
scale into three hierarchical levels consisting of 41 reflective items,
10 subconstructs and three main dimensions, which are similar to the
ones mentioned earlier. The scale was validated through two samples from
the manufacturing industry and further demonstrated nomological validity
\citep{clauss_measuring_2017}. However, both measures are subject to
three significant limitations. Firstly, both measures lack a temporal
component. Consequently, they are inadequate for use in longitudinal
studies or ex-post evaluations of BMI. Secondly, BMI is only measured at
the new-to-the-firm level rather than at the new-to-the-industry or
new-to-the-market level. Thirdly, both measures rely on interviews and
questionnaires, which makes conducting large-scale studies
time-consuming and reliant on the willingness of the companies to
cooperate \citep{clauss_measuring_2017, spieth_business_2016}.

\subsection{BMI and Firm Performance}\label{bmi-and-firm-performance}

A number of studies have examined the relationship between BMI and the
financial performance of a company. These studies differ in terms of
sample size and characteristics, methodology and choice of measures
\citep{foss_fifteen_2017, white_exploring_2022}. Cucculelli \&
Bettinelli \citeyearpar{cucculelli_business_2015} for example
investigate the effect of BMI on sales growth, return on sales (ROS) and
total factor productivity (TFP) in a sample of 376 Italian SMEs in the
years 2000-2010. The results provide support for the hypothesis that BMI
has a positive effect on firm performance, with the effect increasing in
line with the intensity of the innovation. Latifi et al.
\citeyearpar{latifi_business_2021} use a cross-industry sample of 563
European SMEs and provide support for the hypothesis that a company
engaging in BMI will improve its overall firm performance. In their
study, the firm performance is evaluated subjectively in accordance with
the model proposed by Venkatraman \& Ramanujam
\citeyearpar{venkatraman_measurement_1986}. White et al.
\citeyearpar{white_exploring_2022} conducted a meta-analysis based on
the extant BMI literature. After an extensive search of the literature,
they identified 77 studies comprising 26,050 firms from 26 countries and
six continents. They found a positive relationship between BMI and firm
performance, and that this relationship is shaped by factors including
the firm age, industry, the economic and political environment and BMI
characteristics.

\subsection{Text Mining 10-K Filings}\label{text-mining-10-k-filings}

The process of text mining 10-K filings is not a novel concept. Hoberg
\& Phillips \citeyearpar{hoberg_text-based_2016} present a novel
approach to defining industry boundaries. This is achieved through the
parsing of the product descriptions provided by firm 10-K filings and
creating word vectors. Specifically, the authors identify and exclude
proper nouns, which include common words and geographic locations.They
then create word vectors for each firm and year, which enables the
measurement of product similarity over time. In this way the authors
demonstrate shortcomings in the traditional industry classification
systems such as the Standard Industry Classification (SIC) and the North
American Industry Classification System (NAICS), which are not able to
account for temporal changes. The new method is capable of capturing
changes in industry boundaries and competitor sets over time, thereby
providing a dynamic industry classification system. Furthermore, the
authors' findings indicate a correlation between firms' R\&D and
advertising expenditures and an increase in product differentiation.
This underscores the significance of product differentiation. In their
study, Lee \& Hong \citeyearpar{lee_business_2014} examine the evolution
of a firm's BM over time. The authors thus utilize the procedure
proposed by Miner et al. \citeyearpar{miner_practical_2012} to represent
each document as a vector of keywords, which is similar to the approach
utilized by Hoberg \& Phillips \citeyearpar{hoberg_text-based_2016}.
After identifying the Item 1 part of the 10-K filings as the most
crucial part for describing a firm's BM, Lee \& Hong
\citeyearpar{lee_business_2014} filter these for relevant sentences.
Subsequently, the authors construct keyword vectors, which represent the
concept of the BM. Therefore, the evolution of the BM is depicted as the
change in the distribution of keywords over time. Nevertheless, this
approach is not without shortcomings. The authors advocate for a more
robust methodology, such as incorporating multi-word phrases in the
keyword vectors, to enhance the reliability of the approach
\citep{lee_business_2014}.

\subsection{Summary}\label{summary}

BMI is complex concepts which makes it hard to measure. Existing
scale-measures fulfill their purpose but are not sufficient to be used
in large-scale, longitudinal studies. Furthermore, researchers have to
rely on interviews and questionnaires, thus

\begin{itemize}
\tightlist
\item
  BMI is a complex concept which is hard to measure; Measure of Spieth
  fulfills its purpose
\item
  Use Spieth definition for BM and BMI
\item
  There have been attempts to utilize text data from 10-Ks in different
  research fields
\item
  BMI is related to firm performance -\textgreater{} test our measure
  against this assumptions
\end{itemize}

\section{Methods}\label{methods}

\subsection{Assumptions}\label{assumptions}

\subsection{Gemini}\label{gemini}

\subsection{BERT}\label{bert}

\subsection{BERTScore}\label{bertscore}

\subsection{Hierarchical Agglomerative
Clustering}\label{hierarchical-agglomerative-clustering}

\section{Experimental Apparatus}\label{experimental-apparatus}

\subsection{Datasets}\label{datasets}

We collect 10-Ks filings from the digital SEC Database, using the
category ``10-K'' as extraction condition. Since the focus of our study
lies on company's BM, we only use the Item 1 part, since this is the
most crucial part of the 10-K filings for describing the companys BM
\citep{lee_business_2014}.

//Our observations are limited to an intersection of such companies,
which on the one hand has been made available to the SEC since 2001 in a
publicly accessible list of 10.284 companies (Appendix), of which 7590
are listed (on stock exchange). On the other hand, we consider companies
that filed 10-K reports with the SEC between 2017 and 2023
//-\textgreater{} rewrite as step by step, how we got to the final list
of companys

We exclude companies from the financial sector, namely companies with a
SIC Code starting with six. We consider the filings from 2017 to 2023.

TODO - Descriptive Table1 for document length of original filings

\begin{itemize}
\item
  Description of Table1, i.e.~The intersection varies from year to year
  between 3138 and 4929
\item
  Descriptive Table2 for document length of processed filings
\item
  Description of Table2 and the final Dataset
\end{itemize}

\subsection{Pre-Processing}\label{pre-processing}

\begin{itemize}
\tightlist
\item
  everything regarding Gemini in Pre-Processing
\end{itemize}

Since the parts of section I are generally not well structured for
further processing, we use Gemini to summarize the texts with emphasis
on the business description. Based on this data, we create similarity
scores between different companies and the same companies from different
years. Finally, we match these firms with several financial numbers that
we extracted from Eikon/Datastream.

\subsection{Procedure}\label{procedure}

Before generating the final datasets, we had to iteratively coordinate
various steps over several phases in order to generate the data
consistently. The first step was to extract the required company
identifiers (CIKs) from the SEC database.

\section{Results}\label{results}

\section{Discussion}\label{discussion}

\section{Conclusion}\label{conclusion}

\section{Limitations}\label{limitations}

\section{Acknowledgement}\label{acknowledgement}

\begin{itemize}
\tightlist
\item
  Jonathan for IT Support
\item
  Prof.~Kranz for Repo
\end{itemize}

%% begin pandoc before-bib
%% end pandoc before-bib
%% begin pandoc biblio
%% end pandoc biblio
%% begin pandoc include-after
%% end pandoc include-after
%% begin pandoc after-body
%% end pandoc after-body

\end{document}
\endinput
%%
%% End of file `sample-manuscript.tex'.
