\documentclass[manuscript,screen,review]{acmart}


\IfFileExists{upquote.sty}{\usepackage{upquote}}{}
\IfFileExists{microtype.sty}{% use microtype if available
  \usepackage[]{microtype}
  \UseMicrotypeSet[protrusion]{basicmath} % disable protrusion for tt fonts
}{}
\makeatletter
\@ifundefined{KOMAClassName}{% if non-KOMA class
  \IfFileExists{parskip.sty}{%
    \usepackage{parskip}
  }{% else
    \setlength{\parindent}{0pt}
    \setlength{\parskip}{6pt plus 2pt minus 1pt}}
}{% if KOMA class
  \KOMAoptions{parskip=half}}
\makeatother

%%
%% This is file `sample-manuscript.tex',
%% generated with the docstrip utility.
%%
%% The original source files were:
%%
%% samples.dtx  (with options: `manuscript')
%% 
%% IMPORTANT NOTICE:
%% 
%% For the copyright see the source file.
%% 
%% Any modified versions of this file must be renamed
%% with new filenames distinct from sample-manuscript.tex.
%% 
%% For distribution of the original source see the terms
%% for copying and modification in the file samples.dtx.
%% 
%% This generated file may be distributed as long as the
%% original source files, as listed above, are part of the
%% same distribution. (The sources need not necessarily be
%% in the same archive or directory.)
%%
%%
%% Commands for TeXCount
%TC:macro \cite [option:text,text]
%TC:macro \citep [option:text,text]
%TC:macro \citet [option:text,text]
%TC:envir table 0 1
%TC:envir table* 0 1
%TC:envir tabular [ignore] word
%TC:envir displaymath 0 word
%TC:envir math 0 word
%TC:envir comment 0 0
%%
%%
%% The first command in your LaTeX source must be the \documentclass command.


% Options for packages loaded elsewhere
\PassOptionsToPackage{unicode}{hyperref}
\PassOptionsToPackage{hyphens}{url}
\PassOptionsToPackage{dvipsnames,svgnames,x11names}{xcolor}

\IfFileExists{bookmark.sty}{\usepackage{bookmark}}{\usepackage{hyperref}}

%% PANDOC PREAMBLE BEGINS


\providecommand{\tightlist}{%
  \setlength{\itemsep}{0pt}\setlength{\parskip}{0pt}}\usepackage{longtable,booktabs,array}
\usepackage{calc} % for calculating minipage widths
% Correct order of tables after \paragraph or \subparagraph
\usepackage{etoolbox}
\makeatletter
\patchcmd\longtable{\par}{\if@noskipsec\mbox{}\fi\par}{}{}
\makeatother
% Allow footnotes in longtable head/foot
\IfFileExists{footnotehyper.sty}{\usepackage{footnotehyper}}{\usepackage{footnote}}
\makesavenoteenv{longtable}
\usepackage{graphicx}
\makeatletter
\def\maxwidth{\ifdim\Gin@nat@width>\linewidth\linewidth\else\Gin@nat@width\fi}
\def\maxheight{\ifdim\Gin@nat@height>\textheight\textheight\else\Gin@nat@height\fi}
\makeatother
% Scale images if necessary, so that they will not overflow the page
% margins by default, and it is still possible to overwrite the defaults
% using explicit options in \includegraphics[width, height, ...]{}
\setkeys{Gin}{width=\maxwidth,height=\maxheight,keepaspectratio}
% Set default figure placement to htbp
\makeatletter
\def\fps@figure{htbp}
\makeatother

\definecolor{mypink}{RGB}{219, 48, 122}
\makeatletter
\@ifpackageloaded{caption}{}{\usepackage{caption}}
\AtBeginDocument{%
\ifdefined\contentsname
  \renewcommand*\contentsname{Table of contents}
\else
  \newcommand\contentsname{Table of contents}
\fi
\ifdefined\listfigurename
  \renewcommand*\listfigurename{List of Figures}
\else
  \newcommand\listfigurename{List of Figures}
\fi
\ifdefined\listtablename
  \renewcommand*\listtablename{List of Tables}
\else
  \newcommand\listtablename{List of Tables}
\fi
\ifdefined\figurename
  \renewcommand*\figurename{Figure}
\else
  \newcommand\figurename{Figure}
\fi
\ifdefined\tablename
  \renewcommand*\tablename{Table}
\else
  \newcommand\tablename{Table}
\fi
}
\@ifpackageloaded{float}{}{\usepackage{float}}
\floatstyle{ruled}
\@ifundefined{c@chapter}{\newfloat{codelisting}{h}{lop}}{\newfloat{codelisting}{h}{lop}[chapter]}
\floatname{codelisting}{Listing}
\newcommand*\listoflistings{\listof{codelisting}{List of Listings}}
\makeatother
\makeatletter
\makeatother
\makeatletter
\@ifpackageloaded{caption}{}{\usepackage{caption}}
\@ifpackageloaded{subcaption}{}{\usepackage{subcaption}}
\makeatother
%% PANDOC PREAMBLE ENDS

\setlength{\parindent}{10pt}
\setlength{\parskip}{0pt}

\hypersetup{
  pdftitle={The Name of the Title Is Hope},
  pdfauthor={Ben Trovato; G.K.M. Tobin; Lars Thørväld; Valerie Béranger; Aparna Patel; Huifen Chan; Charles Palmer; John Smith; Julius P. Kumquat},
  colorlinks=true,
  linkcolor={blue},
  filecolor={Maroon},
  citecolor={Blue},
  urlcolor={red},
  pdfcreator={LaTeX via pandoc, via quarto}}

%% \BibTeX command to typeset BibTeX logo in the docs
\AtBeginDocument{%
  \providecommand\BibTeX{{%
    Bib\TeX}}}

%% Rights management information.  This information is sent to you
%% when you complete the rights form.  These commands have SAMPLE
%% values in them; it is your responsibility as an author to replace
%% the commands and values with those provided to you when you
%% complete the rights form.
\setcopyright{acmcopyright}
\copyrightyear{2018}
\acmYear{2018}
\acmDOI{XXXXXXX.XXXXXXX}

%% These commands are for a PROCEEDINGS abstract or paper.
\acmConference[Conference acronym 'XX]{Make sure to enter the correct
conference title from your rights confirmation emai}{June 03--05,
2018}{Woodstock, NY}
\acmPrice{15.00}
\acmISBN{978-1-4503-XXXX-X/18/06}

%% Submission ID.
%% Use this when submitting an article to a sponsored event. You'll
%% receive a unique submission ID from the organizers
%% of the event, and this ID should be used as the parameter to this command.
%%\acmSubmissionID{123-A56-BU3}

%%
%% For managing citations, it is recommended to use bibliography
%% files in BibTeX format.
%%
%% You can then either use BibTeX with the ACM-Reference-Format style,
%% or BibLaTeX with the acmnumeric or acmauthoryear sytles, that include
%% support for advanced citation of software artefact from the
%% biblatex-software package, also separately available on CTAN.
%%
%% Look at the sample-*-biblatex.tex files for templates showcasing
%% the biblatex styles.
%%

%%
%% The majority of ACM publications use numbered citations and
%% references.  The command \citestyle{authoryear} switches to the
%% "author year" style.
%%
%% If you are preparing content for an event
%% sponsored by ACM SIGGRAPH, you must use the "author year" style of
%% citations and references.
%% Uncommenting
%% the next command will enable that style.
%%\citestyle{acmauthoryear}


%% end of the preamble, start of the body of the document source.
\begin{document}


%%
%% The "title" command has an optional parameter,
%% allowing the author to define a "short title" to be used in page headers.
\title[Hope]{The Name of the Title Is Hope}

%%
%% The "author" command and its associated commands are used to define
%% the authors and their affiliations.
%% Of note is the shared affiliation of the first two authors, and the
%% "authornote" and "authornotemark" commands
%% used to denote shared contribution to the research.


  \author{Ben Trovato}
  \orcid{1234-5678-9012}
    \author{G.K.M. Tobin}
  
            \affiliation{%
                  \institution{Institute for Clarity in Documentation}
                          \streetaddress{P.O. Box 1212}
                          \city{Dublin}
                                  \country{USA}
                          \postcode{43017-6221}
              }
        \author{Lars Thørväld}
  
            \affiliation{%
                  \institution{The Thørväld Group}
                          \streetaddress{1 Thørväld Circle}
                          \city{Hekla}
                                  \country{Iceland}
                      }
        \author{Valerie Béranger}
  
            \affiliation{%
                  \institution{Inria Paris-Rocquencourt}
                                  \city{Rocquencourt}
                                  \country{France}
                      }
        \author{Aparna Patel}
  
            \affiliation{%
                  \institution{Rajiv Gandhi University}
                          \streetaddress{Rono-Hills}
                          \city{Doimukh}
                                  \country{India}
                      }
        \author{Huifen Chan}
  
            \affiliation{%
                  \institution{Tsinghua University}
                          \streetaddress{30 Shuangqing Rd}
                          \city{Haidian Qu}
                                  \country{China}
                      }
        \author{Charles Palmer}
  
            \affiliation{%
                  \institution{Palmer Research Laboratories}
                          \streetaddress{8600 Datapoint Drive}
                          \city{San Antonio}
                                  \country{USA}
                          \postcode{78829}
              }
        \author{John Smith}
  
            \affiliation{%
                  \institution{The Thørväld Group}
                          \streetaddress{1 Thørväld Circle}
                          \city{Hekla}
                                  \country{Iceland}
                      }
        \author{Julius P. Kumquat}
  
            \affiliation{%
                  \institution{The Kumquat Consortium}
                                  \city{New York}
                                  \country{USA}
                      }
      
\renewcommand{\shortauthors}{Trovato et al.}

%% By default, the full list of authors will be used in the page
%% headers. Often, this list is too long, and will overlap
%% other information printed in the page headers. This command allows
%% the author to define a more concise list
%% of authors' names for this purpose.
%\renewcommand{\shortauthors}{Trovato et al.}
%%  
%% The abstract is a short summary of the work to be presented in the
%% article.
\begin{abstract}
A clear and well-documented \LaTeX~document is presented as an article
formatted for publication by ACM in a conference proceedings or journal
publication. Based on the ``acmart'' document class, this article
presents and explains many of the common variations, as well as many of
the formatting elements an author may use in the preparation of the
documentation of their work.    
\end{abstract}

%%
%% The code below is generated by the tool at http://dl.acm.org/ccs.cfm.
%% Please copy and paste the code instead of the example below.
%%
\begin{CCSXML}
<ccs2012>
 <concept>
  <concept_id>10010520.10010553.10010562</concept_id>
  <concept_desc>Computer systems organization~Embedded systems</concept_desc>
  <concept_significance>500</concept_significance>
 </concept>
 <concept>
  <concept_id>10010520.10010575.10010755</concept_id>
  <concept_desc>Computer systems organization~Redundancy</concept_desc>
  <concept_significance>300</concept_significance>
 </concept>
 <concept>
  <concept_id>10010520.10010553.10010554</concept_id>
  <concept_desc>Computer systems organization~Robotics</concept_desc>
  <concept_significance>100</concept_significance>
 </concept>
 <concept>
  <concept_id>10003033.10003083.10003095</concept_id>
  <concept_desc>Networks~Network reliability</concept_desc>
  <concept_significance>100</concept_significance>
 </concept>
</ccs2012>
\end{CCSXML}

\ccsdesc[500]{Computer systems organization~Embedded systems}
\ccsdesc[300]{Computer systems organization~Redundancy}
\ccsdesc{Computer systems organization~Robotics}
\ccsdesc[100]{Networks~Network reliability}

%%
%% Keywords. The author(s) should pick words that accurately describe
%% the work being presented. Separate the keywords with commas.
\keywords{datasets, neural networks, gaze detection, text tagging}


%%
%% This command processes the author and affiliation and title
%% information and builds the first part of the formatted document.
\maketitle

\setlength{\parskip}{-0.1pt}

\section{Introduction}\label{introduction}

Business model innovation (BMI) is a key activity to maintain
competitiveness and even gain a competitive advantage
\citep[@teece\_business\_2018]{pucihar_drivers_2019}. It is therefore no
surprise that the interest in BMI and methods of measuring it has grown
rapidly over the last twenty years. Researchers have recently called for
a BMI measurement instrument that is more comprehensive and advanced
than already existing ones \citep{huang_review_2023}. The scale
developed by \citep{spieth_business_2016} provides managers and
practitioners with a measurement index for business model
innovativeness. This measurement model only validates applicability of
BMI theory \citep{huang_review_2023} and is insufficient for
longitudinal studies \citep{clauss_measuring_2017}. Hence, this measure
is not adequate for a time series analysis of BMI. Furthermore, it
refers only to BMI as new-to-the-firm and is not able to grasp BMI in
the sense of new-to-the-industry and new-to-market.

\begin{itemize}
\item
  Solution approach (both after we actually are final with approach and
  have (some) results)
\item
  Results
\end{itemize}

We contribute in multiple ways. First we apply a more modern text-mining
technique than Lee and Hong (2014) and Hoberg and Phillips (2016) to
extract the business model from 10-K filings. Second, we propose a new
measure for BMI which is sufficient for longitudinal studies and able to
measure BMI on a new-to-the-industry level.

\section{Related Work}\label{related-work}

\subsection{Business Model Innovation}\label{business-model-innovation}

In spite of the growing interest in BMI and the increasing number of
theoretical and empirical studies in this field, the research of BMI is
still in a prelaminar state \citep{huang_review_2023}. Consequently,
this study focuses on literature pertaining to the measurement of BMI.
\citep{spieth_business_2016} identify three core dimensions a company's
business model is comprised of: its value proposition, its value
creation architecture and its revenue model logic. Based on this, BMI is
conceptualised as a change that is new-to-the-firm in at least one of
these dimensions. Furthermore, \citep{spieth_business_2016} introduce a
measurement model to evaluate these three dimensions of BMI. They
develop an index by first specifying the contents, followed by a
specifiaction of the indicators and assessing their content validity,
assessing the indicators collinearity and finally assessing the external
validity. In total, twelve indicators for measuring the innovativeness
of the business model were identified by a literature review and
entering in a dialog with practitioners. The external validity of the
formative indicators was successfully validated by questioning 200
strategy and innovation management experts.
\citep{clauss_measuring_2017} follows a very similar approach. After
specifying the domain and dimensionality of BMI trough a literature
research, the author divides the scale into three hierarchical levels
consisting of 41 reflective items, 10 subconstructs and three main
dimensions, which are similar to the ones mentioned earlier. The scale
is validated through two large-scale samples from the manufacturing
industry and further demonstrated nomological validity
\citep{clauss_measuring_2017}.

\subsection{Text Mining}\label{text-mining}

\subsection{Company size}\label{company-size}

\section{Methods}\label{methods}

\subsection{Assumptions}\label{assumptions}

\section{Experimental Apparatus}\label{experimental-apparatus}

\section{Results}\label{results}

\section{Discussion}\label{discussion}

\section{Conclusion}\label{conclusion}

\section{Limitations}\label{limitations}

\section{Acknowledgement}\label{acknowledgement}

\bibliographystyle{ACM-Reference-Format}
\bibliography{bibliography.bib}

%% begin pandoc before-bib
%% end pandoc before-bib
%% begin pandoc biblio
%% end pandoc biblio
%% begin pandoc include-after
%% end pandoc include-after
%% begin pandoc after-body
%% end pandoc after-body

\end{document}
\endinput
%%
%% End of file `sample-manuscript.tex'.
